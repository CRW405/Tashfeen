\documentclass{homework}
\author{Wiyninger, Caleb}
\class{CSCI 2114: Tashfeen's Data Structures}
\date{\today}
\title{Homework 7}
\address{%
  Oklahoma City University, %
  Petree College of Arts \& Sciences, %
  Computer Science%
}

\acmfonts

\begin{document} \maketitle

\question Solve any fourteen problems over at
\url{https://projecteuler.net}.

\begin{enumerate}
  \item Your program should take no more than a minute to solve any given
        problem. You may use either Java or Python for this. You are only
        allowed to use the standard libraries in either language.
  \item All your programs must compile and run from the command line by
        either using \texttt{javac} and \texttt{java} commands, e. g.,
        \begin{verbatim}
javac Tashfeen1.java
java Tashfeen1
\end{verbatim}
        Or,
        \begin{verbatim}
python3 Tashfeen1.py
\end{verbatim}
  \item You should only have \textit{ one file for each problem}. Name
        each file starting with your last name followed by the problem
        number. E. g., if I solved the first Project Euler problem, I'll
        name it \texttt{Tashfeen1.java}.
  \item Your program should print nothing but the number that solves the
        problem.
  \item Include your answer as a comment at the top of each source file.
        See listing \ref{exmp} for an example.
\end{enumerate}

% \lstinputlisting[
%   language={java},
%   caption={A solution to the Project Euler's first problem as an example.},
%   label=exmp]
% {./code/Tashfeen1.java}

\section{Submission Instructions}

Please \textit{replace the professor's last name} with yours
wherever appropriate.

Setup a GitHub repository by the name
\texttt{f23csci2114hw7tashfeen} \footnote{E. g., replace the
  professor's last name by yours here.}. It should contain,
\begin{verbatim}
f23csci2114hw7tashfeen
├── scTashfeen.png
├── Tashfeen1.java
├── Tashfeen2.java
├── Tashfeen3.java
├── Tashfeen4.java
├── Tashfeen5.java
├── Tashfeen6.java
├── Tashfeen7.java
├── Tashfeen8.java
├── Tashfeen9.java
├── Tashfeen10.java
├── Tashfeen11.java
├── Tashfeen12.java
├── Tashfeen13.java
└── Tashfeen14.java
\end{verbatim}

Submit all the files to the online classroom and comment a link to
your Github repository. The file called \texttt{scTashfeen.png} is a
screenshot showing all the problems you solved on Project Euler and
your username. See figure \ref{sc} for example.

\img<sc>[0.65]{Screenshot of solved project Euler problems and username.}{media/sc.png}

\end{document}
