\documentclass{homework}
\author{Wiyninger, Caleb}  % Uncomment with your name
\class{MATH 3503: Tashfeen's Discrete Mathematics}
\date{\today}  % Uncomment with your semester
\title{Homework 2}
\address{
  Computer Science, %
  Petree College of Arts \& Sciences, %
  Oklahoma City University
}

\begin{document} \maketitle

\question Please read chapter 2 of Chartrand et al. and write a couple sentences about a topic/example/concept that you found difficult or interesting and why?

\begin{sol}
    I like how the book put the concepts of the different types of number in terms of sets such as $\N \subset \Z$. This made some concepts easier to understand.
\end{sol}

\question How many elements are in $\mathcal{P}(A)$ if $A = \{n \in \Z: |n| \leq 5\}$?

\begin{sol}
11 integers \smallbreak
$2^{11}$ elements
\end{sol}

\question Let $A = \{0, \{0\}, \{0, \{0\}\}\}$.
\begin{enumerate}[label=(\alph*)]
    \item Determine which of the following are elements of $A: 0, \{0\}, \{\{0\}\}$.

    % solution goes here
    \begin{sol}
        0, \{0\}
    \end{sol}

    \item Determine $|A|$

    % solution goes here
    \begin{sol}
    3
    \end{sol}
    
    \item Determine which of the followin+g are subsets of $A: 0, \{0\}, \{\{0\}\}$.

    % solution goes here
    \begin{sol}
    \{0\}, \{\{0\}\}
    \end{sol}
    
    For (d)-(i), determine the indicated sets.
    \item $\{0\} \cap A$.

    % solution goes here
    \begin{sol}
    \{0\}
    \end{sol}
    
    \item $\{\{0\}\} \cap A$.

    % solution goes here
    \begin{sol}
    \{\{0\}\}
    \end{sol}
    
    \item $\{\{\{0\}\}\} \cap A$.

    % solution goes here
    \begin{sol}
        empty
    \end{sol}
    
    \item $\{0\} \cup A$.

    % solution goes here
    \begin{sol}
        \{0, \{0\}, \{0, \{0\}\}\}
    \end{sol}
    
    \item $\{\{0\}\} \cup A$.

    % solution goes here
    \begin{sol}
    \{0, \{0\}, \{0, \{0\}\}\}
    \end{sol}
    
    \item $\{\{\{0\}\}\} \cup A$.

    % solution goes here
    \begin{sol}
    \{0, \{0\}, \{0, \{0\}\}, \{\{0\}\}\}
    \end{sol}
\end{enumerate}
\question For two sets $A$ and $B$ of real numbers, the set $A \cdot B$ is defined by,
\[
    A \cdot B = \{ab: a \in A, b \in B\}.
\]
Determine each of the following sets.
\begin{enumerate}
    \item $A \cdot B$ for $A = \{\frac{1}{2}, 1, \sqrt{2}\}$ and $B = \{\sqrt{2}, 2, 4\}$.

    % solution goes here
    \begin{sol}
    $\{\frac{\sqrt2}{2}, 1, 2, \sqrt{2}, 4, 2\sqrt{2}, 4\sqrt{2}\}$
    \end{sol}
    
    \item $\R \cdot \R$.

    % solution goes here
    \begin{sol}
    $\R$
    \end{sol}
    
    \item $\R \cdot C$ where $C \subseteq \R$ with $|C| = 2$.

    % solution goes here
    \begin{sol}
    $\R$
    
    \end{sol}

\end{enumerate}
\question For $A = \{1, 2\}, B = \{-1, 0, 1\}$ and the universal set $U = \{-2, -1, 0, 1, 2\}$, determine
\begin{enumerate}[label=(\alph*)]
    \item $A \cup B$.

    % solution goes here
    \begin{sol}
    \{-1, 0, 1, 2\}
    \end{sol}
    
    \item $A \cap B$.

    % solution goes here
    \begin{sol}
    \{1\}
    \end{sol}
    
    \item $A - B$.

    % solution goes here
    \begin{sol}
    \{2\}
    \end{sol}
    
    \item $\overline{B}$.

    % solution goes here
    \begin{sol}
    \{-2, 2\}
    
    \end{sol}
    
    \item $A \times B$.

    % solution goes here
    \begin{sol}
    \{(1,-1), (1,0), (1,1), (2,-1), (2,0), (2,1)\}
    \end{sol}
\end{enumerate}

\question Give examples of three sets $A, B$ and $C$ such that
\begin{enumerate}[label=(\alph*)]
    \item $A \subseteq B \not\subset C$.

    % solution goes here
    \begin{sol}
        A = \{1\} \smallbreak
        B = \{1, 2\} \smallbreak
        C = \{3\} \smallbreak
    \end{sol}
    
    \item $A \subseteq B, B \in C$ and $A \cap C = \nil$.

    % solution goes here
    \begin{sol}
        A = \{1\} \smallbreak
        B = \{1,2\} \smallbreak
        C = \{\{1,2\}, 3\} \smallbreak
    \end{sol}
    
    \item $A \in B, A \subset B$ and $A \not\subseteq C$.

    % solution goes here
    \begin{sol}
        A = \{1\} \smallbreak
        B = \{\{1\}, 1, 2\} \smallbreak
        C = \{3\} \smallbreak
    \end{sol}
    
    \item $A \in B, A \not\subseteq B$ and $B \in C$.

    % solution goes here
    \begin{sol}
        A = \{1\} \smallbreak
        B = \{\{1\}, 2\} \smallbreak
        C = \{\{\{1\},2\}, 1\} \smallbreak
    \end{sol}
\end{enumerate}
\end{document}
