\documentclass{homework}
\author{Wiyninger, Caleb}  % Uncomment with your name
\class{MATH 3503: Tashfeen's Discrete Mathematics}
\date{Spring 2023}  % Uncomment with your semester
\title{Homework 4}
\address{
  Computer Science, %
  Petree College of Arts \& Sciences, %
  Oklahoma City University
}

\begin{document} \maketitle

\question Please read chapter 5 of Chartrand et al. and write a couple sentences about a topic/example/concept that you found difficult or interesting and why?

% solution here

\question The following are relations on the set $\R$ of real numbers. Which of the properties reflexive, symmetric and transitive does each relation below possess?
\begin{enumerate}[label=(\alph*)]
	\item $x \ R_1 \ y \text{ if } |x - y| \leq 1$.
	      % solution here
        \begin{sol}
            \begin{enumerate}
                \item Reflexive since $x-x=0\leq1$
                \item Symmetric since $|x-y|=|y-x|$
                \item Not Transitive since it could be $|y-z|>1$
            \end{enumerate}
        \end{sol}
	\item $x \ R_2 \ y \text{ if } y \leq 2x + 1$.
	      % solution here
        \begin{sol}
            \begin{enumerate}
              \item Not Reflexive since x could be $< -1$
              \item Not Symmetric since x could be $> 2y+1$
              \item Not Transitive since y could be $> 2z+1$
            \end{enumerate}
        \end{sol}
	\item $x \ R_3 \ y \text{ if } y = x^2$.
	      % solution here
        \begin{sol}
            \begin{enumerate}
              \item Not Reflexive since $x\neq x^2$
              \item Not Symmetric since y may not equal $x^2$
              \item Not Transitive since z may not equal $y^2$
            \end{enumerate}
        \end{sol}
	\item $x \ R_4 \ y \text{ if } x^2 + y^2 = 9$.
	      % solution here
        \begin{sol}
            \begin{enumerate}
             \item Not Reflexive since $2x^2$ may not equal 9
              \item Symmetric since order of addition does not matter
              \item Not Transitive since $y^2 + z^2$ may not equal 9
            \end{enumerate}
        \end{sol}
\end{enumerate}

\question Let $S = \{1,2,3,4,5,6,7\}$. The relation
\[
	R = \{(1, 1),(1, 3),(1, 4),(2, 2),(3, 1),(3, 3),(3, 4),(4, 1),(4, 3),(4, 4),(5, 5),(5, 7),(6, 6),(7, 5),(7, 7)\}
\]
on $S$ is an equivalence relation. Determine the distinct equivalence classes.

% Solution here
\begin{sol}
  \[[1]=\{1,3,4\}\]
  \[[2]=\{2\}\]
  \[[5]=\{5,7\}\]
  \[[6]=\{6\}\]
\end{sol}

\question Give an example of an equivalence relation $R$ on the set $A = \{1, 2, 3, 4, 5, 6, 7\}$ with $\mathcal{P}$ the set of equivalence classes such that the following four properties are satisfied:

\begin{enumerate}
	\item $|\mathcal{P}| = 3$,
	      % solution here
        \begin{sol}
          \[\{1,2\}\]
          \[\{3,4\}\]
          \[\{5,6,7\}\]
        \end{sol}
	\item There exists no set $S \in \mathcal{P}$ such that $|S| = 3$,
	      % solution here
        \begin{sol}
          \[\{1,2\}\]
          \[\{3,4\}\]
          \[\{5,6\}\]
          \[\{7\}\]
        \end{sol}
	\item $3 \ \mathord{\not\mathrel{R}} \ 4 \text{ but } 3 \ R \ 5$,
	      % solution here
        \begin{sol}
          \[\{1,2\}\]
          \[\{3,5\}\]
          \[\{4,6\}\]
          \[\{7\}\]
        \end{sol}
	\item There exists a set $T \in \mathcal{P}$ such that $1, 7 \in T$.
	      % solution here
        \begin{sol}      
          \[\{1,2,7\}\]
          \[\{3,5\}\]
          \[\{4,6\}\]
        \end{sol}
\end{enumerate}

\question Let $A = \{1, 2, 3\}$, $B = \{1, 2, 3, 4, 5\}$ and $C = \{1, 2, 3, 4\}$. Also let $f : A \ra B$ and $g : B \ra C$, where $f = \{(1, 4), (2, 5), (3, 1)\}$ and $g = \{(1, 3), (2, 3), (3, 2), (4, 4), (5, 1)\}$,
\begin{enumerate}[label=(\alph*)]
    \item Determine $(g \circ f )(1), (g \circ f )(2)$ and $(g \circ f )(3)$.
	      % solution here
    \item Determine $g \circ f$.
	      % solution here
\end{enumerate}

\question Each of the following is a function from $\N \times \Z$ to $\Z$. Which of these are onto?
\begin{enumerate}[label=(\alph*)]
    \item $f(a, b) = 2a + b$
	      % solution here
    \item $f(a, b) = b$
	      % solution here
    \item $f(a, b) = 2^ab$
	      % solution here
    \item $f(a, b) = |a|-|b|$
	      % solution here
    \item $f(a, b) = a+10$
	      % solution here
\end{enumerate}

\question Let $f, g$ and $h$ be functions from $\R$ to $\R$ defined by $f(x) = e^x, g(x) = x^3$ and $h(x) = 3x$ for each $x \in \R$. Determine each of the following:
\begin{enumerate}[label=(\alph*)]
    \item $(g \circ f )(x).$
	      % solution here
    \item $(f \circ g)(x).$
	      % solution here
    \item $(h \circ f )(x).$
	      % solution here
    \item $(f \circ h)(x).$
	      % solution here
    \item a composition of functions that results in $e^{3x^3}$
	      % solution here
    \item a composition of functions that results in $3e^{x^3}$
	      % solution here
\end{enumerate}

\question Let $f: \R \ra \R$ be defined by $f(x) = 5x + 3$.
\begin{enumerate}
    \item Show that $f$ is one-to-one.
	      % solution here
    \item Show that $f$ is onto.
	      % solution here
    \item Find $f^{-1}(x)$ for $x \in \R$.
	      % solution here
\end{enumerate}

\question Prove that the function $f:\R-\{3\} \ra \R - \{1\}$ defined by $f(x)=\frac{x}{x-3}$ is bijective.

% solution here
\end{document}
