\documentclass{homework}
\author{Wiyninger, Caleb}  % Uncomment with your name
\class{MATH 3503: Tashfeen's Discrete Mathematics}
\date{Spring 2023}  % Uncomment with your semester
\title{Homework 3}
\address{
  Computer Science, %
  Petree College of Arts \& Sciences, %
  Oklahoma City University
}

\begin{document} \maketitle

\question Please read chapters 3 and 4 of Chartrand et al. and write a couple sentences about a topic/example/concept that you found difficult or interesting and why?

\question Consider the following quantified statement: For every real number $x$, there exists a positive real number $y$ such that $y < x^2$.

\begin{enumerate}[label=(\alph*)]
	\item Express this quantified statement in symbols.
	\begin{sol}
		\[
			\forall x \in \R. \exists y \in \R.  (y > 0 \land y < x^2)
		\]
	\end{sol}
	\item Express the negation of this quantified statement in symbols.
	\begin{sol}	
		\begin{align*}
			\exists x \in \R. \forall y \in \R. (y \leq 0 \vee y \geq x^2)
		\end{align*}
	\end{sol}
	\item Express the negation of this quantified statement in words. \\
		  \begin{sol}
			For a real number x, there does not exist a positive real number y such that $y < x^2$
		  \end{sol}
\end{enumerate}

\question Prove that if $r$ and $s$ are rational numbers, then $r - s$ is a rational number.

\begin{sol}
	\begin{proof} if $(r \in \R \land s \in \R)$ then $(r - s) \in \R$
		\begin{enumerate}
			\item Assume $r = \frac{x}{y} \land s = \frac{j}{k}$
			\item Rational numbers are numbers that can be expressed as a fraction
			\item Then, $ (r - s) = (\frac{x}{y} - \frac{j}{k}) = \frac{xk-yj}{yk}$
			\item since (r - s) can be written as a function then (r - s) is a rational number
			\item Therefore, $(r - s) \in \Q$
		\end{enumerate}
	\end{proof}
\end{sol}

\question Let $x$ and $y$ be integers. Prove that if $x + y \geq 9$, then either $x \geq 5$ or $y \geq 5$.
\begin{sol}
	\begin{proof}[Proof by Contrapositive] $x \in \Z \land y \in \Z. x + y \geq 9. x \geq 5 \lor y \geq 5$
		\begin{enumerate}
			\item Assume $(x \in \Z \land y \in \Z) \land (x < 5 \land y < 5)$
			\item Then $(x \leq 4 \land y \leq 4)$
			\item Therefore, $(x + y) \leq (4 + 4) < 9$
		\end{enumerate}
	\end{proof}
\end{sol}

\question Let $m$ and $n$ be two integers. Prove that $mn$ and $m + n$ are both even if and only if $m$ and $n$ are both even.

\begin{sol}
	\begin{proof} $(m \in \Z \land n \in \Z)$. mn is even $\land$ m + n is even if m is even $\land$ n is even \\
		Assume $m = 2x \land n = 2y$ \\
		Then $mn = 4xy \land (m + n) = (2x + 2y)$ \\
		Therefore, $mn \land (m + n)$ are even because they can be written in terms of 2\\
\smallbreak
		Assume $mn = 4xy \land (m + n) = (2x + 2y)$ \\
		Then $m = 2x \land n = 2y$\\
		Therefore, $m \land n$ are even becuase they can be written in terms of 2
	\end{proof}
\end{sol}

\question Disprove: Let $A, B$ and $C$ be sets. If $A \cup B = A \cup C$, then $B = C$.

\begin{sol}
	\begin{enumerate}
		\item Assume $A = \{1, 2\}, B = \{2, 3\}, C = \{3\}$
		\item Then $A \cup B = A \cup C$, but $B \neq C$
		\item Therefore If $A \cup B = A \cup C$, then $B = C$ is not true
	\end{enumerate}
\end{sol}

\question Prove that if $a$ and $b$ are positive real numbers, then $\sqrt{a} + \sqrt{b} \neq \sqrt{a+b}$.
\begin{sol}
	\begin{proof}
		\begin{enumerate}
			\item Assume $a = x^2 \land b = x^2$
			\item Then $\sqrt{a} + \sqrt{b} = \sqrt{x^2} + \sqrt{x^2} = x + x = 2x$ \\
			$\sqrt{a+b} = \sqrt{2x^2} = x\sqrt{2}$
			\item Therefore $\sqrt{a} + \sqrt{b} \neq \sqrt{a+b}$ can be written as $2x \neq x\sqrt{2}$ which is true
		\end{enumerate}
	\end{proof}
\end{sol}

\question Let $r \geq 2$ be an integer. Prove that $1 + r + r^{2} + \cdots + r^{n} = \frac{r^{n+1}-1}{r-1}$ for every positive integer $n$.

\begin{sol}
\[
	r \in \Z, \geq 2 \quad
	n \in \Z, > 0 \quad
	1 + r + r^2 + \cdots + r^n = \frac{r^{n+1} - 1}{r - 1}
\]
\begin{proof}[Proof by Induction] % original statement.

    \textbf{Basis:} Take $n\in\{...\}$ then we have,
    \[
        % base case 1  ,\quad
        % base case 2
		\frac{r^{1+1}-1}{r-1} = \frac{r^2 - 1}{r - 1} = \frac{(r-1)(r+1)}{r-1} = r+1 \quad
		\frac{r^{0+1}-1}{r-1} = 1
    \]
    \textbf{Inductive Hypothesis:} Assume for some positive integer $n$,
    \[
        % state your inductive hypothesis
		n = k \quad \frac{r^{k+1}-1}{r-1}
    \]
    \textbf{Inductive Step:} We show the bellow by induction on $n$,
    \begin{align*}
		(1+r+r^2+\cdots+r^k)+r^{k+1} = \\
		\frac{r^{k+1}-1}{r-1} + r^{k+1} = \\ 
		\frac{r^{k+1}-1}{r-1} + \frac{r^{k+1}(r-1)}{r-1} = \\
		\frac{r^{k+1}-1+(r^{k+1}r - r^{k+1})}{r-1} = \\
		\frac{r^{k+1}-1+(r^{k+1}r - r^{k+1})}{r-1} = \\
		\frac{r^{k+1}-1+(r^{k+2} - r^{k+1})}{r-1} = \\
		\frac{r^{k+2}-1}{r-1} =\\
		\frac{r^{(k+1)+1}-1}{r-1} \\
	\end{align*}
        % state the inductive step to be argued
		
    % use your inductive hypothesis to show your inductive step.

    Then by induction on $n$ we showed that % reiterate the original statement. \qedhere
	\[
		1 + r + r^2 + \cdots + r^n = \frac{r^{n+1} - 1}{r - 1}
	\]

\end{proof}
\end{sol}

\question Prove that $\frac{1}{\sqrt{1}} + \frac{1}{\sqrt{2}} + \cdots + \frac{1}{\sqrt{n}} > \sqrt{n+1} $ for every integer $n\geq 3$.

\begin{sol}
	\begin{proof}
		Base Case
		\[
			\frac{1}{\sqrt{1}} + \frac{1}{\sqrt{2}} + \frac{1}{\sqrt{3}} > \sqrt{4}
		\]
		Hypothesis
		\[
			\frac{1}{\sqrt{1}} + \frac{1}{\sqrt{2}} + \cdots + \frac{1}{\sqrt{k}} > \sqrt{k+1}
		\]
		Inductive Step
		\[
			(\frac{1}{\sqrt{1}} + \frac{1}{\sqrt{2}} + \cdots + \frac{1}{\sqrt{k}}) + \frac{1}{\sqrt{k+1}} > \sqrt{k+1} + \frac{1}{\sqrt{k+1}}
		\]
		\[
		\sqrt{k+1} + \frac{1}{\sqrt{k+1}} = \frac{\sqrt{k+1}\sqrt{k+1}}{\sqrt{k+1}} + \frac{1}{\sqrt{k+1}} = \frac{k+2}{\sqrt{k+1}}
		\]
		\bigbreak
		\[
			\frac{k+2}{\sqrt{k+1}} > \sqrt{(k+1)+1} = 
		\]
		\[
			(\frac{k+2}{\sqrt{k+1}})^2 > (\sqrt{(k+1)+1} )^2 =
		\]
		\[
			\frac{(k+2)^2}{k+1} > k+2 =
		\]
		\[
			(k+2)^2 > (k+2)(k+1)
		\]
		\[
			k+2>k+1
		\]
		By proof induction on n we show that
		\[
			\frac{1}{\sqrt{1}} + \frac{1}{\sqrt{2}} + \cdots + \frac{1}{\sqrt{n}} > \sqrt{n+1} 
		\]
	\end{proof}
\end{sol}

\question A sequence $a_{1}, a_{2}, a_{3},\cdots$ is defined recursively by $a_{1} = 3$ and $a_{n} = 2a_{n-1} + 1$ for $n\geq 2$.
\begin{enumerate}[label=(\alph*)]
	\item Determine $a_{2}, a_{3}, a_{4},$ and $a_{5}$.
	\begin{sol}
		\[a_2 = 7\]
		\[a_3 = 15\]
		\[a_4 = 31\]
		\[a_5 = 63\]
	\end{sol}
	\item Based on the values obtained in (a), make a guess for a formula for $a_{n}$ for every positive integer $n$ and use induction to verify that your guess is correct.
	\begin{sol}
		\begin{proof}
			formula
		    \[
				a_{0} = 1 \quad \text{and} \quad a_{n} = 2a_{n-1} + 1
			\]
			Base
			\[
				a_{1} = 3
			\]
			hypothesis
			\[
				a_k = 2a_{k-1}+1
			\]
			inductive step
			\[
				a_{k+1} = 2(2a_{k-1}+1) + 1
			\]
			\[
				a_{k+1} = 4a_{k-1}+3
			\]	
			\[
				a_5 = 4a_3 + 3
			\]
			\[
				63 = 4(15) + 3
			\]
			\[
				63 = 63
			\]
			Therefore
			\[
				a_k = 2a_{k-1}+1
			\]
		\end{proof}
	\end{sol}
\end{enumerate}

\question In Example 4.36, we saw that $n^{th}$ Fibonacci number $F_{n} \leq 2^{n}$. Prove that $F_{n}\leq (\frac{5}{3})^{n}$ for every positive integer $n$.

\begin{sol}
	\begin{proof}
		Base
		\[
			F_1 \leq \frac{5}{3}^1 = 1 \leq \frac{5}{3}
		\]
		Hypothesis
		\[
			F_k \leq \frac{5}{3}^k
		\]
		Inductive step
		\[
			F_{k+1} \leq \frac{5}{3}^{k+1}
		\]
		\[
			F_{k+1}\leq\frac{5}{3}F_k\leq\frac{5}{3}^{k+1}
		\]
		Therefore
		\[
			F_{k+1} \leq \frac{5}{3}^{k+1}
		\]
	\end{proof}
\end{sol}

\question A sequence \{$a_{n}$\} is defined recursively by $a_{1} = 5$, $a_{2} = 7$ and ${a_{n} = 3a_{n-1}-2a_{n-2}-2}$ for $n\geq3$. Prove that $a_{n} = 2n + 3$ for every positive integer $n$.
\begin{sol}
	\begin{proof}
		Base
		\[
			a_3 = 9
		\]
		Hypothesis
		\[
			a_k = 2k + 3
		\]
		Inductive step
		\[
			a_{k+1} = 2(k+1) + 3 = 2k+5
		\]
		\[
			2n+5 = {3a_{n-1}-2a_{n-2}-2}
		\]
		\[
			2(3)+5 = 
		\]
	\end{proof}
\end{sol}

\end{document}
